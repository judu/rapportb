\documentclass[french, titlepage, 10pt, a4paper]{article}
\usepackage{xunicode}
\usepackage{fontspec}
\usepackage[frenchb]{babel}
\usepackage{graphicx}
\usepackage{listings}
\usepackage[T1]{fontenc}
\usepackage{kpfonts}
\usepackage[usenames,dvipsnames]{color}
\usepackage{zed}

\renewcommand{\nobreakspace}{\nobreak\ }

\author{Julien \textsc{Durillon} \and Alexandre \textsc{Garnier}}
\title{Spécification d'un gestionnaire de services}

\begin{document}

\maketitle

\section{Introduction}

Dans le cadre du module de \emph{Construction formelle du logiciel en B} du
Master ALMA de l'Université de Nantes, nous avons eu à concevoir la
spécification formelle en B d'un système des gestion de services au sein d'un
OS.\\

Il s'agit plus précisément d'une architecture basée sur un ensemble de services
accessibles par un nombre variable de processus, selon que ces services soient
en accès exclusif ou inclusif.
L'accès aux services exclusifs devra être basé sur des profils de processus, de
sorte que seuls les processus dont le profil est reconnu par un service pourront
utiliser celui-ci.

Ce faisant, on distingue trois concepts principaux au sein du projet, à savoir
les services, les processus et les profils.
Dès lors, dans notre travail de conception puis de spécification, nous avons eu
pour objectif principal de refléter cette telle répartition.\\

Dans la suite, après une analyse plus avancée du sujet, nous présenterons la
spécification formelle des différents aspects attendus du logiciel, ainsi que
les différents raffinements et implantations opérées.

Enfin, nous discuterons de la solution finale, des problèmes encontrés dans sa
conception, et des avantages et inconvénients que nous avons pu mettre à jour
quant à l'utilisation de B, et plus spécifiquement d'AtelierB.

\section{Analyse}

\subsection{Les éléments du système}

Pour pouvoir répondre aux conditions du système, nous devons manipuler les
concepts suivants:

\begin{itemize}
  \item processus: un processus identifié par un numéro;
  \item service: un service est utilisé par un processus;
  \item profile: un processus possède un profil; un service peut filtrer les
    processus autorisés via leur profil.
\end{itemize}

\subsection{Première itération sur l'architecture}

Dans la première version, nous avons commencé par créer une machine pour gérer
les processus, leurs profils et les services.
Nous nous sommes rendus compte que cette version était lourde et peu adaptée.

\subsection{Deuxième version : modularisation}

Les concepts \emph{processus} et \emph{profile} sont intimement liés.
Nous allons donc séparer la conception en deux parties: d'un côté la gestion des
processus et de leur profil, de l'autre la gestion des services.

\subsubsection{ProcManager}

La machine ProcManager définit le concept de \emph{processus} et celui de
\emph{profile}.

\subsection{Machine composite}

Pour gérer les différents éléments, une machine Système doit être faite.
Elle va définir les actions haut-niveau qui utiliseront les processus et les
services.

\section{Spécification}

\subsection{Ensembles et relations: diagrammes d'Euler-Venn}

Dans la spécification formelle du projet, nous avons d'emblée veillé à ce que
les concepts de services, processus et profils soient formalisés sous forme
d'ensembles, respectivement \emph{services}, \emph{processus} et
\emph{profiles}.

\subsubsection{Gestionnaire de processus}

Dans la gestion des processus, apparaît la notion de profil d'un processus.
Tout processus est lié à un profil, cependant plusieurs processus peuvent
partager un même profil.
En outre, un profil n'a pas de raison d'exister si aucun processus ne lui est
lié.

\begin{figure}[htb]
  \centering
  \includegraphics[width=0.4\textwidth]{proc_profile.png}
  \caption{$proc\_profile: processus \rightarrow profiles$}
  \label{fig:proc_profile}
\end{figure}

Dès lors, nous avons formalisé une fonction totale de \emph{processus} vers
\emph{profiles}, telle que l'illustre la figure \ref{fig:proc_profile}.

\subsubsection{Gestionnaire de services}

Dans le cadre du gestionnaire de services, les relations entre ensembles
deviennent dès lors plus nombreuses, dans la mesure où sont réutilisées ici les
concepts de processus et de profils afin de mieux satisfaire à la modularité
attendue dans la gestion de services.

\begin{enumerate}

  \item Type d'un service:

    Tout service a un type d'accès, définissant si l'accès à ce service est
    contraint ou libre.
    Un service contraint devra spécifier les profils de processus pouvant
    l'utiliser.

    \begin{figure}[htb]
      \centering
      \includegraphics[width=0.4\textwidth]{type_service.png}
      \caption{$type\_service: services \rightarrow TYPE$}
      \label{fig:type_service}
    \end{figure}

    La figure \ref{fig:type_service} montre la fonction liant un service à son
    type.

  \item Accès d'un service:

    L'accès à un service permet de spécifier si plusieurs processus peuvent
    faire appel à ce service en même temps (accès inclusif) ou non (accès
    exclusif).

    \begin{figure}[htb]
      \centering
      \includegraphics[width=0.4\textwidth]{acces_service.png}
      \caption{$acces\_service: services \rightarrow ACCES$}
      \label{fig:acces_service}
    \end{figure}

    La figure \ref{fig:acces_service} montre la fonction liant un service à son
    accès.

  \item État d'un service:

    L'état d'un service permet de gérer la disponibilité ou non de celui-ci.
    Le service sera dès lors respectivement actif ou inactif.

    \begin{figure}[htb]
      \centering
      \includegraphics[width=0.4\textwidth]{etat_service.png}
      \caption{$etat\_service: services \pfun ETAT$}
      \label{fig:etat_service}
    \end{figure}

    La figure \ref{fig:etat_service} montre la fonction liant un service à son
    état.

  \item Profils que reconnaît un service contraint:

    Un service contraint est lié à un ou plusieurs profils de processus qui
    pourront dès lors l'utiliser.

    \begin{figure}[htb]
      \centering
      \includegraphics[width=0.4\textwidth]{serv_profiles.png}
      \caption{$serv\_profiles: serv\_contraints \rightarrow FIN(PROFILES)$}
      \label{fig:serv_profiles}
    \end{figure}

    La figure \ref{fig:serv_profiles} montre la fonction liant un service
    contraint à ses profils de processus.

  \item Souscription d'un processus à un service:

    Un processus doit pouvoir souscrire à un service, afin de pouvoir l'utiliser
    ensuite.
    Plusieurs processus peuvent souscrire à un même service, et un processus
    peut souscrire à plusieurs services.

    \begin{figure}[htb]
      \centering
      \includegraphics[width=0.4\textwidth]{serv_proc.png}
      \caption{$serv\_subscribed: services \leftrightarrow PROCESSUS$}
      \label{fig:serv_subscribed}
    \end{figure}

    La figure \ref{fig:serv_subscribed} montre la relations liant des services
    à des processus leur ayant souscrit.

  \item Utilisation d'un service par un processus:

    Un processus doit dès lors être en mesure d'utiliser un service auquel il a
    souscrit.
    Pour peu que l'accès à un service soit inclusif, plusieurs processus peuvent
    l'utiliser en même temps, et un processus peut utiliser plusieurs services à
    la fois.\\

    La relation $serv\_binded: services \leftrightarrow PROCESSUS$ est
    dès lors similaire à la figure \ref{fig:serv_subscribed}.

\end{enumerate}

\subsection{Opérations des machines}

\section{Organisation du travail}
Plusieurs aspects nous ont poussés à travailler ensemble sur le projet: la
gestion des projets dans AtelierB ne nous permettait pas un partage simple des
fichiers de projets; la méthode B nous étant encore assez étrangère, nous avons
préféré travailler ensemble sur les spécifications et sur le développement pour
assurer une meilleure qualité.

Dans la partie processus, il doit être possible de créer et de supprimer un
processus, d'y associer un profil.

Dans la partie service, il doit être possible de déclarer un service, de
l'activer, d'y associer les profils autorisés, d'inscrire des processus à un
service.

\section{Conclusion}

\end{document}

