\documentclass[french, 11pt, a4paper]{article}
\usepackage{xunicode}
\usepackage{fontspec}
\usepackage[frenchb]{babel}


\author{Julien \textsc{Durillon} \and Alexandre \textsc{Garnier}}
\title{Spécification d'un gestionnaire de services}

\begin{document}
\maketitle



\section{Introduction}



\section{Analyse}

    \subsection{Les éléments du système}
        Pour pouvoir répondre aux conditions du système, nous devons manipuler les concepts suivants :

        \begin{itemize}
            \item processus : un processus identifié par un numéro
            \item service : un service est utilisé par un processus
            \item profile : un processus possède un profil ; un service peut filtrer les processus autorisés via leur profil
        \end{itemize}


    \subsection{Première itération sur l'architecture}
        Les concepts \emph{processus} et \emph{profile} sont intimement liés.
        Nous allons donc séparer la conception en deux parties : d'un côté la
        gestion des processus et de leur profil, de l'autre la gestion des
        services.

        Dans la partie processus, il doit être possible de créer et de supprimer
        un processus, d'y associer un profil.

        Dans la partie service, il doit être possible de déclarer un service, de
        l'activer, d'y associer les profils autorisés, d'inscrire des processus à un service.




\end{document}

